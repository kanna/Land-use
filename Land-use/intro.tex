\subsection*{Introduction}

The oceans covering more than 70\% of the Earth's surface are not only
a sink for Carbon, but also a major repository of the global capacity
to store heat from anthropogenic sources. The impact of rising sea
surface temperatures not only impacts marine life, but also human life
deep within terrestrial ecosystems with the increasing mass of heated
air transported from across the oceans over coastal boundaries, to
inland areas. The end result of such transport has led to floods, high
surf, hurricanes and tornados among other extreme events, not just in
coastal zones, but well inland. Even in urban areas, extreme events
have had a range of impacts with secondary effects from the
oceans. After effects of Hurricane Ida in Louisiana in 2021 led to an
excess of 3.91''/hour of rain in New York City, flooding basements in
Queens killing 11. \com{ajit: This is per hour - the point being that it is
  an extreme precipitation event with a large volume falling in a very
  short time.  The total rain across the region over the day was more
  like 6-8 inches}.

\subsection{Alternative Policy Need Section to replace two sections
  above}

Science and technology have made rapid strides in the prediction of
extreme weather events to a point where advanced monitoring and
modeling can be operationalized by emergency responders. By
integrating these advancements into planning tools, emergency managers
can stress-test the preparedness of emergency response and make
decisions on asset allocation and placement before disasters occur,
with the overall goal of minimizing impact on communities. Such
advancements can also incorporate the distributional effects of
disasters on communities, helping emergency response organizations and
others make decisions that reduce such inequities.


The integration of coastal riverine, land use and ocean dynamics
coupled with a logistical course of action (COA) will enable agencies
such as FEMA, Coast Guard and the Army Corps of Engineers to do ’what-
if-analyses’ to enhance preparedness. To do so, we need not just
models but hard and soft observations from a range of embedded and
on-the-ground sensors on mobile and static platforms (autonomous or
not) coupled with high resolution models with a predictive capability
which can be run effectively and quickly anytime.

\subsection*{Proposed Solution}

To limit loss of life and alleviate human suffering as a result of
such events, technology can and should play a prominent role. Higher
resolution ocean, atmospheric, estuarine and land use models can be
coupled to provide ways to enable 'what-if' analysis for policy makers
and other stakeholders. Increased model skill has resulted in better
predictions, but this is pre-medicated on having higher resolution
data at fine scales. Absent such high-resolution data, such models
cannot be effective in their forecasting precision. 

\begin{wrapfigure}{r}{0.45\textwidth}
  \centering
  \includegraphics[scale=0.5]{fig/trioka.jpg}
  \caption{Predictive capacity go hand in hand with ocean/land use
    models derived from high resolution sensors from in-situ robotics.}
  \label{fig:tri}
\end{wrapfigure}

We believe the use of AI-driven networked mobile robotic platforms
carrying a range of payloads in space, aerial, surface, terrestrial
and underwater domains can provide the necessary capability for
precisely such data driven high-resolution prediction \com{ajit: I am not
  convinced that this particular configuration is necessary or even
  appropriate given that you need to deploy just before a storm}. They
can be targeted to be in the 'right place at the right time' as a
result of models being able to bound their own uncertainty. As a
result, the 'three poles' which enable such a responsive capacity
leading to coastal resilience include observations from hardware
robotic platforms, as well as Machine Learned methods mining troves of
remote sensing data all in high resolution, coupled ocean and
atmospheric models which can not only model the dynamic ocean and its
surface, but also interact with atmospheric phenomenon which provide
oceanographic forcing and finally the predictions for future
ocean/atmospheric state which can impact human habitation inland and
on shore (Fig. \ref{fig:tri}).

When these are combined with terrestrial land use, ecological and
hydrological models, a system-of-system then can provide stakeholders,
especially policymakers with a capability for 'what-if' scenario
generation at \textbf{block-by-block level} of urban environments,
which in turn can aid in the planning for extreme events for agencies
like FEMA especially in urban environments. 

Such tools can then be used in a number of ways, foremost being in
stress-testing the preparedness of emergency responders and disaster
management agencies, while also providing strategic advice on asset
allocation and placement. Another would be in using such ensemble
modeling for Monte-Carlo type repeat simulations to come up with the
most likely set of scenarios which could help in building planning
resilience.
