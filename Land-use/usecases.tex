\subsection*{Use Cases}

We envision a range of use cases for such an approach involving
integrated modeling of terrestrial and coastal ocean models driven by
high resolution data from a range of sensors on static and mobile
platforms: 

\begin{description}

  
\item[Impacts of nature on the built environment] Where built structures
  can be safe from oceanic or terrestrial events has typically relied on
  '100 year' events of the past. Yet the changing nature of the
  environment has predicated that we make informed 'guesses' of where to
  build human structures for habitation, entertainment or business. Such
  integrative tools with a high-resolution (block-by-block) capability
  can allow for fine-grained analysis of how natural events can conspire
  to make such human structures be impacted by the changing natural
  environment. 

\item[Flooding in urban centers] With 'what-if' analysis and high
  resolution models with digital elevation maps (DEMs) a block by block
  analysis of flooding impacts can be carefully mapped leading to
  informed decisions on what kinds of urban search and rescue operations
  are required. These same analyses could also inform city-planning and
  lead to enforcement actions towards saving lives, much as was desired
  with the effects of Ida in NYC in 2021.

\item[Strengthening of levees] As Hurricane Katarina in 2005 showed, the
  impact of storm surges with the after effects of a hurricane can have
  direct and disastrous consequences of levee failures. Systems such as
  those proposed can enable decision-makers to run 'what-if' analyses
  with integrated hydrological and ocean models to targeted assessments
  of such critical infrastructure.

\item[Inland flooding] Increasing energy contained in atmospheric storms
  including cyclones and hurricanes can lead to substantial downpours
  inland where the topology of the land and lack of adequate drainage
  facilities could innudate urban areas locally or with flash floods
  downstreams of rivers and streams at capacity. Such an integrated
  system which can predict outcomes of such natural events at fine
  scales leading to better flooding estimates as against relying on
  notions of a '100 year flood' which were conceived well before the
  rapid changes in climate. 

\item[Monitoring Coastal Pollution] The Deepwater Horizon incident
  showed a direct need to understand coastal domains to predict the
  transport of oil spills and measure its impact on the coastal
  communities in Louisiana, Texas, Missisipi and Florida. The GofM
  domain is complex with a range of estuarine, delta and riverine
  features with a range of topology on land and underwater, which made
  landfall prediction of dense crude oil challenging. High resolution
  coastal and terrain models coupled with hydrology could allow
  policymakers to run multiple simulations towards a more accurate
  predictive capability to fortify sensitive coastal areas and
  communities well in advance, if not actively during the onset of an
  event.

\end{description}  